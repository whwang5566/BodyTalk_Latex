\begin{abstractzh}
在網路快速發展之下,數以百萬計的人透過網路分享照片,撰寫網誌,或是與其他人遊玩遊戲。然而,語言仍然阻礙了人們聯繫、社交以及合作。
為了了解語言障礙對於合作遊戲體驗的影響,我們進行了一個12人的使用者測試,透過使用者遊玩三款知名的合作遊戲,我們得到的結果是,無共通語言的受測者遊玩體驗獲得了較高的挫折感以及較低的遊戲樂趣。因此我的提出透過肢體語言來跨越合作遊戲中的語言障礙,我們的系統使用了Kinect來偵測使用者的姿勢並對應到遊戲中的虛擬角色。共48位使用者參與了我們遊戲原型的使用者測試,我們發現了肢體語言可以提升遊戲樂趣以及降低挫折感,並加強合作體驗,尤其對於無共通語言的使用者來說效果更是明顯。此外,有79\%的受測者偏好有肢體語言的溝通方式。
\end{abstractzh}

\begin{abstracten}

The rapid growth of the Internet has enabled billions of people to share photos, to blog, and to play games with each other. However, languages remain a barrier for people to connect, socialize, and cooperate around the world. 
We conducted a 12-person user study to understand how language barriers affect cooperative experiences through 3 popular cooperative games. Results showed that participants without common languages rated their experience as significantly more frustrating and less fun. We propose using body language to transcend language barriers for cooperative games. Our system, BodyTalk, uses Kinect sensors to track users' postures and share them as avatars in real-time with other users over the Internet. Our 48-person user study using our prototype game showed that cooperative experience with body language was more fun and less frustrating, and co-experience was improved for all participants, especially for those without common languages. 
Also, 79\% of the participants preferred having body language communication. 

\end{abstracten}

\begin{comment}
\category{I2.10}{Computing Methodologies}{Artificial Intelligence --
Vision and Scene Understanding} \category{H5.3}{Information
Systems}{Information Interfaces and Presentation (HCI) -- Game-based
Interaction.}

\terms{Design, Human factors, Performance.}

\keywords{Cooperative game, Body language, Game design, Communication Pattern, Kinect, Human computer interaction.}
\end{comment}
