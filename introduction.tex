\chapter{Introduction}

Screen sharing protocols have been developed for a long time such as Remote Desktop Protocol (RDP)~\cite{RDP} and Virtual Network Computing (VNC)~\cite{VNC} which uses Remote Frame Buffer protocol (RFB)~\cite{RFB}. There are also lots of softwares like vaiants of VNC software and Remote Desktop Services (RDS)~\cite{RDS}. Screen sharing softwares are used to control remote computer or help remote people.

On mobile devices, screen sharing is also called screen mirroring. The screen mirroring is not like screen sharing on PC. Screen mirroring only projects the content on mobile device to another screen like TV. No user input events are transmitted through the connection. There are increasing numbers of mobile devices that support screen mirroring. The iOS devices with version 5 or above support AirPlay Mirroring. User could project the display content of his iOS device. Some Android devices with version 4.4.1 or above might be able mirror display content to Chromecast. Miracast is also a screencasting standard that more and more Android devices support.

Screen mirroring connection could be established using Wi-Fi or Wi-Fi Direct. People using screen mirroring to share display content to others. While playing games, people using screen mirroring to project display content to a bigger screen to improve the experience. The display that mobile device is mirroring to may be a public display, e.g., sharing in office. Alternatively, it may be a private display, e.g., playing game in the dorm. The focus of this paper is the information leakage when using screen mirroring on private display.

With increased proliferation devices that support screen mirroring, we notice some different in display content between PCs and mobile devices. The differences may cause some security issues. They can be categorized into three groups:

First, on mobile devices, all user events generate feedbacks for user experience, e.g., change the size or color when pressing button. Some credential might leak if attacker could intercept the screen content. Although user input events do not transmit over screen mirroring, the feedbacks do show on the screen, e.g., the PIN lock feedback on iOS or Android or the software keyboard feedback. [圖]. The attacker may implement a key logger using the feedbacks of software keyboard if he could intercept all mirroring data.

Second, the password fields in mobile devices show the last character by default. The reason is that the software keyboards are small and do not have tactile feedback. The last character helps user to make sure what he typed. All login windows, e.g. Google Play, App store, and login form in mobile website have this feature by default. The attack could easily retrieve plaintext password from the mirroring data by this feature.

Third, the notifications in mobile devices may leak some credentials. The system notification of incoming SMS or email shows partial text on screen. It provides attacker to retrieve some privacy information like billing address or order number. Moreover, attacker might use this to skip the multi-factor authentication. Take Google's two-step verification~\cite{GoogleTwoStep} for example: The attacker who got the password of victim account could not login the account which protected by two-step verification. He need the unique code that sent to victim phone via SMS to login victim account. If the attacker could intercept the mirroring data, he could know the unique code from system SMS notification, and then compromised the target account [圖].

The problems above do not exist in PC. On PC, password fields are always protected, and no feedbacks are generated when users have keystrokes. If the screen sharing connection do not transmit keystroke data, the attacker couldn't retrieve it from screen data. In mobile devices, we need to handle mirroring data more carefully. We also implement a man-in-the-middle (MITM) attack to demonstrate how worse can it be when the mirroring data is intercepted.
